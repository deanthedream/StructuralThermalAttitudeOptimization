\documentclass{article}
\usepackage{graphicx}
\usepackage{url}
\usepackage{amsmath}

\begin{document}

\title{Fitting Shapes To A Point Cloud}
\author{Dean R Keithly}

\maketitle

\begin{abstract}
The abstract text goes here.
\end{abstract}

\section{Introduction}



\section{Fitting 2-D Circle to X,Y Points}
Our goal is to approximate the center and radius of a set of points which lie on a circle.
We are given an arbitrary set of points in the 2D plane which are defined in the orthogonal X and Y coordinate plane.

We find the bisection point between any two adjacent points on the circle using
\begin{align*}
x_{bi} = \frac{x_i+x_{i+1}}{2},\\
y_{bi} = \frac{y_i+y_{i+1}}{2}.
\end{align*}

We can calculate their slope by
\begin{align*}
dx_i = x_{i+1} - x_i,\\
dy_i = y_{i+1} - y_i,\\
m_i = -\frac{dx_i}{dy_i}.
\end{align*}
We now have a point half-way between two points and the slope of a line perpendicular to the point to point connecting line so we can calculate the y-axis intercept
\begin{align*}
b_i = y_{bi} - m_i*x_{bi}.
\end{align*}

This line should go through the circle center but we do not know exactly where the center point is.
We will therefore repeat this process for all points.
We can trivially calculate the X point of intercept between any two of these lines using 
\begin{align*}
x_{int}(i,j) = \frac{b_i-b_j}{m_j-m_i}\\
\end{align*}
and then calculate the $y_{int}$
\begin{align*}
y_{int}(i,j) = x_{int}(i,j)*m_i + b_i.
\end{align*}





\begin{equation}
    \label{simple_equation}
    \alpha = \sqrt{ \beta }
\end{equation}

\subsection{Subsection Heading Here}
Write your subsection text here.

%\begin{figure}
%    \centering
%    \includegraphics[width=3.0in]{myfigure}
%    \caption{Simulation Results}
%    \label{simulationfigure}
%\end{figure}

\section{Conclusion}
Write your conclusion here.

\end{document}